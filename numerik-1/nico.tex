
\subsection{Allgemein}
\begin{itemize}
\item $x_0$ "`sinnvoll"' wählen
\item direkte Verfahren: Exaktes Ergebnis nach endlichen Anzahl Rechenschritten
\item iterative Verfahren: Lösung, die gegen Ergebnis konvergiert
\end{itemize}

\subsection{Bisektionsverfahren, S. 18}

\begin{itemize}
\item Für Nullstelle: 2 Punkte wählen, $x_1$ mit $y_1 > 0$ und $x_2$ mit $y_2 < 0$
\item Allgemein: Für a-Stelle: 2 Punkte wählen, $x_1$ mit $y_1 > a$ und $x_2$ mit $y_2 < a$
\end{itemize}
\[x3 = \frac{x_1 + x_2}{2}\]

\begin{itemize}
\item $if(f(x3)) > 0$
\begin{itemize}
\item $x_1$ = $x_3$
\end{itemize}
\item else
\begin{itemize}
\item $x_2$ = $x_3$
\end{itemize}
\end{itemize}

\subsection{Fixpunktiteration, S. 21, S.23}
\begin{itemize}
\item Ausgangsformel umstellen nach $x = $, das ergibt $F(x)$
\end{itemize}
\[x_{n+1} = F(x_n)\]

Nachschauen, ob Fixpunkt an- oder abstossend ist:
\begin{itemize}
\item konvergiert (anziehend): $|F'(x)| < 1$
\item abstossend: $|F'(x)| > 1$
\end{itemize}


\subsubsection{Fehlerabschätzung}

\begin{itemize}
\item $x_n$ = iterierte Wert
\item a-priori: Anzahl benötigte Schritte für Fehlergenauigkeit
\item a-posterori: Fehler nach n Iterierungen bestimmen
\end{itemize}

a-priori:
\[|x_n - \overline{x}| \leq{} \frac{a^n}{1 - a} \cdot |x_1 - x_0|\]

a-posterori:
\[|x_n - \overline{x}| \leq{} \frac{a}{1 - a} \cdot |x_n - x_{n-1}|\]

\subsection{Newtonverfahren, S. 25}

\[x_{n+1} = x_n - \frac{f(x_n)}{f'(x_n)}\]

\subsection{Vereinfachte Newtonverfahren, S. 25}

\[x_{n+1} = x_n - \frac{f(x_n)}{f'(x_0)}\]

\subsection{Sekantenverfahren}

\[x_{n+1} = x_n - \frac{x_n-x_{n-1}}{f(x_n) - f(x_{n-1})} \cdot f(x_n)\]


\subsection{Gauss-Algorithmus, S.33}
\begin{itemize}
\item Grundidee:
\begin{itemize}
\item Obere Dreiecksmatrix
\item Einsetzen von unten nach oben
\end{itemize}
\item direktes Verfahren
\end{itemize}

Vorgehen obere Dreiecksmatrix:
\begin{itemize}
\item Gehe durch die Spalten von \textbf{1} bis \textbf{n-1}
\begin{itemize}
\item Wenn Diagonalelement = 0, dann nichts machen
\item Wenn andere Zeile dort 0 hat, Zeile vertauschen
\item Ansonsten:
\begin{itemize}
\item $j = spalte + 1, ..., n$
\item $z_j := z_j - \frac{a_{jspalte}}{a_{spaltespalte}} \cdot z_{spalte}$
\end{itemize}
\end{itemize}
\end{itemize}
Einsetzen:

\subsection{Vektor- und Matrixnormen, S. 45}
Dort zu finden :-)

\subsection{Matrizenzerlegung (A = L + D + R) sowie $D^{-1}$, S.50, S.52}
Super einfach: Die Matrix wird zerschnitten und
\begin{itemize}
\item L = links untere
\item D = Diagonalmatrix
\item R = rechts obere
\end{itemize}

$D^{-1}$ ist trivial: Kehrwert für jedes Element: $1/a_{ii}$

\subsection{Jacobi-Verfahren, S. 51}

\[D{x^{(n+1)}} = - (L+R) \cdot x^{(n)} + b\]
\[{x^{(n+1)}} = -D^{-1} (L+R) \cdot x^{(n)} + D^{-1} b\]

$D^{-1}$, L, R aus vorherigem Kapitel, b muss gegeben sein,
$-D^{-1}$ ist die Negierung. Berechnung in Einzelteilen:

\[
x_j^{(m+1)} = \frac{1}{a_{jj}} \left( b_j - \displaystyle\sum\limits_{k=1, k \neq j}^n a_{jk}x_{k}^{(m)} \right) \quad j = 1, ..., n
\]
