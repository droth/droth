\subsection{Bisektionsverfahren, S. 18}

\begin{itemize}
\item Für Nullstelle: 2 Punkte wählen, $x_1$ mit $y_1 > 0$ und $x_2$ mit $y_2 < 0$
\item Allgemein: Für a-Stelle: 2 Punkte wählen, $x_1$ mit $y_1 > a$ und $x_2$ mit $y_2 < a$
\end{itemize}
\[x3 = \frac{x_1 + x_2}{2}\]

\begin{itemize}
\item $if(f(x3)) > 0$
\begin{itemize}
\item $x_1$ = $x_3$
\end{itemize}
\item else
\begin{itemize}
\item $x_2$ = $x_3$
\end{itemize}
\end{itemize}

\subsection{Fixpunktiteration, S. 21, S.23}
\begin{itemize}
\item Ausgangsformel umstellen nach $x = $, das ergibt $F(x)$
\item $x_0$ "`sinnvoll"' wählen
\end{itemize}
\[x_{n+1} = F(x_n)\]

Nachschauen, ob Fixpunkt an- oder abstossend ist:
\begin{itemize}
\item konvergiert (anziehend): $|F'(x)| < 1$
\item abstossend: $|F'(x)| > 1$
\end{itemize}


\subsubsection{Fehlerabschätzung}

\begin{itemize}
\item $x_n$ = iterierte Wert
\item a-priori: Anzahl benötigte Schritte für Fehlergenauigkeit
\item a-posterori: Fehler nach n Iterierungen bestimmen
\end{itemize}

a-priori:
\[|x_n - \overline{x}| \leq{} \frac{a^n}{1 - a} * |x_1 - x_0|\]

a-posterori:
\[|x_n - \overline{x}| \leq{} \frac{a}{1 - a} * |x_n - x_{n-1}|\]
