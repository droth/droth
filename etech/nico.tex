\subsection{Strom/Spannungsquellen}
\begin{itemize}
\item ideale ... = Bleibt immer gleich
\end{itemize}
% ------------------------------------------------------------------------------ 
\subsection{Dioden/LEDs}
\begin{itemize}
\item Dioden == LEDs (für uns)
\item LEDs == die Dinger mit Pfeilen dran
\item Können Strom sperren (wenn Pfeil entgegen Stromrichtung)
\item "`Haben keinen Widerstand"' (Strom kann nur via Knotengleichung bestimmt werden)
\item Ideale Spannungsquelle
\end{itemize}
% ------------------------------------------------------------------------------ 
\subsection{Quellentransformation}
\[UQ = \frac{(U1 * \sum{Widerstaende_{unten}})}{\sum{alle Widerstaende}}\]


\[RQ = \frac{(\sum{Widerstaende_{unten}} * \sum{Widerstaende_{oben}})}{\sum{alle Widerstaende}}\]

\subsection{Quellentransformation mit Dioden}
\[UQ = \frac{((U1-\sum{UF_{Dioden}} * \sum{Widerstaende_{unten}})}{\sum{alle Widerstaende}} + \sum{UF_{Dioden}}\]
\[RQ = \frac{(\sum{Widerstaende_{unten}} * \sum{Widerstaende_{oben}})}{\sum{alle Widerstaende}}\]
% ------------------------------------------------------------------------------ 
\subsection{Bi-Polar-Transistor}
\[I_e = I_b * (B+1)\]
\[I_e = I_c + I_b <=> I_c = I_e - I_b\]

\begin{itemize}
\item Basis-Kreis und Emitterkreis können unabhängig voneinander berechtnet werden
\item $I_{c}$ ist eine ideale Stromquelle
\end{itemize}
Bestimmung von $I_b$ durch "`alle Spannungen durch Widerstände"':
\[U_Q = U_{RQ} + U_{basis-widerstaende} + U_{be} + U_{emmiter-widerstaende}\]
Umstellen, Ströme und Widerstände Einsetzen:
\[U_Q - U_{be} = I_b * R_q + I_b * basis-widerstaende + I_e * emmiter-widerstaende\]
$I_e$ nach $I_b$ auflöse:
\[U_Q - U_{be} = I_b * R_q + I_b * basis-widerstaende + I_b * (B+1) * emmiter-widerstaende\]
Nach $I_b$ auflöse:
\[\frac{U_Q - U_{be}}{R_q + basis-widerstaende + ((B+1) * emmiter-widerstaende)} = I_b\]

Wenn noch Dioden im Spiel sind: Ganz normal wieder von der Spannung abziehen.
Danach $I_{c}$  und $I_{e}$ bestimmen, wie oben angegeben.
